\documentclass[slidestop,compress,mathserif,xcolor=svgnames,table]{beamer}

\usepackage[spanish]{babel}
\usepackage[utf8]{inputenc} % Ambos para solucin de asuntos de idioma.
\usepackage{amsmath, amsfonts, amssymb} % Matemticas varias.
\usepackage{longtable} % Tabla larga necesaria para el apndice de simbologa.
%\usepackage{multimedia}
\usepackage{movie15}
\usepackage{enumitem}

\setitemize{leftmargin=*}
\setitemize[1]{label=\tiny$\blacksquare$}
\setitemize[2]{label=\tiny$\bullet$}
\setitemize[3]{label=\tiny$\bullet$}

% --- Arreglos varios para la inclusion de imgenes
\usepackage{subfigure}
\DeclareGraphicsExtensions{.png,.jpg,.pdf,.mps,.gif,.bmp}

% ================= CONFIGURACIN DEL BEAMER ====================
% ===============================================================
\mode<presentation>{
% \usepackage{beamerthemeclassic}
% \setbeameroption{show notes}
% \setbeameroption{hide notes}
\setbeameroption{notes on second screen}
% \setbeamercovered{transparent}

% \usetheme{Warsaw}
% \usetheme{Frankfurt}
% \usetheme{Berlin}
 \usetheme{Madrid}
% \usetheme{Antibes}

% \usecolortheme{rose}
% \usecolortheme{lily}
% \usecolortheme{seahorse}
% \usecolortheme{fly}
\usecolortheme[RGB={70,110,45}]{structure}	% verde primaveral
% \usecolortheme[RGB={100,12,20}]{structure}	% rojo sangre
% \usecolortheme[RGB={30,40,30}]{structure}	    % Luto
% \usecolortheme[RGB={160,230,70}]{structure}	% verde manzana Grand-Smith
 
% \usefonttheme[onlymath]{serif}
\usefonttheme{professionalfonts}

\setbeamerfont{frametitle}{family=\rmfamily,shape=\scshape,size=\normalsize}
%\setbeamertemplate{frametitle}[default][center]
}
% Make a TOC appear before every section
\AtBeginSection[]
{\begin{frame}
  \tableofcontents[currentsection]
 \end{frame}
}
\AtBeginNote{Notas:\par}


% ========================= COMANDOS ============================
% ===============================================================
\newcommand{\ket}[1]{|#1\rangle}
\newcommand{\bcol}{\begin{columns}[T]}
\newcommand{\ecol}{\end{columns}}
\newcommand{\col}[1][0.5]{\column{#1\textwidth}}
\newenvironment{itemize*}%
  {\begin{itemize}%
    \setlength{\topsep}{10pt}%
    \setlength{\itemsep}{0pt}%
    \setlength{\parskip}{10pt}}%
  {\end{itemize}
}

% =================== DATOS DEL DOCUMENTO =======================
% ===============================================================
\author{Gonzalo Gutiérrez, Matas Tailanián}
\date{19 de Octubre de 2012}
\title{\textit{Seguimiento de Tempo}}
\subtitle{Procesamiento digital de senales de audio\\Curso 2012}



% ===============================================================
% ===============================================================
% ================ COMIENZO DEL DOCUMENTO =======================
% ===============================================================
% ===============================================================



\begin{document}

% ========================= FRAME ===============================
% ===============================================================
\begin{frame}
  \titlepage
\end{frame}

% ========================= FRAME ===============================
% ===============================================================
\begin{frame}
\frametitle{Resumen}
\tableofcontents
\end{frame}


\section{Seccion 1}
\subsection{Subseccion 1}
% ========================= FRAME ===============================
% ===============================================================
\begin{frame}
\frametitle{Titulo de la diapo}
	
\end{frame}


\section{Seccion 2}
% ========================= FRAME ===============================
% ===============================================================
\begin{frame}
\frametitle{Ttulo}

\end{frame}


\section{Referencias}
% ===============================================================
% ========================= FRAME ===============================
% ===============================================================
\begin{frame}
\frametitle{Referencias:}
\begin{thebibliography}{99}
\begin{small}
\bibitem{bib:algo} Ejemplo. Aca pongo la primera referencia
\vspace{-5pt}
\bibitem{bib:otra_cosa}  Aca pongo otra referencia
\vspace{-5pt}
\bibitem{bib:y_asi} Y asi sucesivamente
\end{small}
\end{thebibliography}
\end{frame}


\frame{\frametitle{blocs}

\begin{block}{title of the bloc}
bloc text
\end{block}

\begin{exampleblock}{title of the bloc}
bloc text
\end{exampleblock}


\begin{alertblock}{title of the bloc}
bloc text
\end{alertblock}
}



\end{document}



% ================ FRAME TPICO CON FIGURA ======================
% ===============================================================

\begin{frame}
\frametitle{Qu es un Parcial?}

\begin{columns}

\column{2in}
\vspace{-25pt}
\begin{figure}[h!]
  \begin{center}
  \includegraphics[width=.85\textwidth]{./Pics/Espectrograma_ejemplo.jpg}
  \end{center}
  \vspace{-10pt}
  \caption{Espectrograma}
  \label{fig:espectrograma_ejemplo}
\end{figure}

\column{2in}
\begin{itemize}
	\item \textbf{Item 1}:\pause
  	\begin{itemize*}
	  \item subitem 1 \pause
	  \item subitem 2 \pause
	  \item subitem 3 \pause
	\end{itemize*}
	\item \textbf{Item 2}
\end{itemize}

\end{columns}
\end{frame}




% ======================== FIGURAS ==============================
% ===============================================================

\begin{figure}[h!]
	\centering
	\includegraphics[width=0.75\textwidth]{./Pics/		.eps}
	\caption{		}
	\label{fig:		}
\end{figure}

\begin{figure} [h!]
  \centering
  \subfloat[caption 1]{\label{fig:		}
  		\includegraphics[width=0.45\textwidth]
  			{./Pics/		.eps}}
  \subfloat[caption 2]{\label{fig:		} 
  		\includegraphics[width=0.45\textwidth]
  			{./Pics/ 		.eps}}
  \caption{Caption general}
  \label{fig:	label general	}
\end{figure}

\begin{wrapfigure}{l}{0.6\textwidth}
  \vspace{-20pt}
  \begin{center}
    \includegraphics[width=0.45\textwidth]
    	{./Pics/		.eps}
  \end{center}
  \vspace{-20pt}
  \caption{		}
  \label{ 		}
  \vspace{-10pt}
\end{wrapfigure}

\begin{figure} [h!]
  \begin{center}
    \begin{tabular}{cc}
      \resizebox{50mm}{!}
      	{\includegraphics{./Pics/ 	.eps}} &
      \resizebox{50mm}{!}
      	{\includegraphics{./Pics/	.eps}} \\
      \resizebox{50mm}{!}
      	{\includegraphics{./Pics/	.eps}} &
      \resizebox{50mm}{!}
      	{\includegraphics{./Pics/	.eps}} \\
    \end{tabular}
    \caption{ 		}
    \label{ 		}
  \end{center}
\end{figure}
